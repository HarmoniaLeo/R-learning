% Options for packages loaded elsewhere
\PassOptionsToPackage{unicode}{hyperref}
\PassOptionsToPackage{hyphens}{url}
%
\documentclass[
]{article}
\usepackage{lmodern}
\usepackage{amssymb,amsmath}
\usepackage{ifxetex,ifluatex}
\ifnum 0\ifxetex 1\fi\ifluatex 1\fi=0 % if pdftex
  \usepackage[T1]{fontenc}
  \usepackage[utf8]{inputenc}
  \usepackage{textcomp} % provide euro and other symbols
\else % if luatex or xetex
  \usepackage{unicode-math}
  \defaultfontfeatures{Scale=MatchLowercase}
  \defaultfontfeatures[\rmfamily]{Ligatures=TeX,Scale=1}
\fi
% Use upquote if available, for straight quotes in verbatim environments
\IfFileExists{upquote.sty}{\usepackage{upquote}}{}
\IfFileExists{microtype.sty}{% use microtype if available
  \usepackage[]{microtype}
  \UseMicrotypeSet[protrusion]{basicmath} % disable protrusion for tt fonts
}{}
\makeatletter
\@ifundefined{KOMAClassName}{% if non-KOMA class
  \IfFileExists{parskip.sty}{%
    \usepackage{parskip}
  }{% else
    \setlength{\parindent}{0pt}
    \setlength{\parskip}{6pt plus 2pt minus 1pt}}
}{% if KOMA class
  \KOMAoptions{parskip=half}}
\makeatother
\usepackage{xcolor}
\IfFileExists{xurl.sty}{\usepackage{xurl}}{} % add URL line breaks if available
\IfFileExists{bookmark.sty}{\usepackage{bookmark}}{\usepackage{hyperref}}
\hypersetup{
  hidelinks,
  pdfcreator={LaTeX via pandoc}}
\urlstyle{same} % disable monospaced font for URLs
\usepackage[margin=1in]{geometry}
\usepackage{color}
\usepackage{fancyvrb}
\newcommand{\VerbBar}{|}
\newcommand{\VERB}{\Verb[commandchars=\\\{\}]}
\DefineVerbatimEnvironment{Highlighting}{Verbatim}{commandchars=\\\{\}}
% Add ',fontsize=\small' for more characters per line
\usepackage{framed}
\definecolor{shadecolor}{RGB}{248,248,248}
\newenvironment{Shaded}{\begin{snugshade}}{\end{snugshade}}
\newcommand{\AlertTok}[1]{\textcolor[rgb]{0.94,0.16,0.16}{#1}}
\newcommand{\AnnotationTok}[1]{\textcolor[rgb]{0.56,0.35,0.01}{\textbf{\textit{#1}}}}
\newcommand{\AttributeTok}[1]{\textcolor[rgb]{0.77,0.63,0.00}{#1}}
\newcommand{\BaseNTok}[1]{\textcolor[rgb]{0.00,0.00,0.81}{#1}}
\newcommand{\BuiltInTok}[1]{#1}
\newcommand{\CharTok}[1]{\textcolor[rgb]{0.31,0.60,0.02}{#1}}
\newcommand{\CommentTok}[1]{\textcolor[rgb]{0.56,0.35,0.01}{\textit{#1}}}
\newcommand{\CommentVarTok}[1]{\textcolor[rgb]{0.56,0.35,0.01}{\textbf{\textit{#1}}}}
\newcommand{\ConstantTok}[1]{\textcolor[rgb]{0.00,0.00,0.00}{#1}}
\newcommand{\ControlFlowTok}[1]{\textcolor[rgb]{0.13,0.29,0.53}{\textbf{#1}}}
\newcommand{\DataTypeTok}[1]{\textcolor[rgb]{0.13,0.29,0.53}{#1}}
\newcommand{\DecValTok}[1]{\textcolor[rgb]{0.00,0.00,0.81}{#1}}
\newcommand{\DocumentationTok}[1]{\textcolor[rgb]{0.56,0.35,0.01}{\textbf{\textit{#1}}}}
\newcommand{\ErrorTok}[1]{\textcolor[rgb]{0.64,0.00,0.00}{\textbf{#1}}}
\newcommand{\ExtensionTok}[1]{#1}
\newcommand{\FloatTok}[1]{\textcolor[rgb]{0.00,0.00,0.81}{#1}}
\newcommand{\FunctionTok}[1]{\textcolor[rgb]{0.00,0.00,0.00}{#1}}
\newcommand{\ImportTok}[1]{#1}
\newcommand{\InformationTok}[1]{\textcolor[rgb]{0.56,0.35,0.01}{\textbf{\textit{#1}}}}
\newcommand{\KeywordTok}[1]{\textcolor[rgb]{0.13,0.29,0.53}{\textbf{#1}}}
\newcommand{\NormalTok}[1]{#1}
\newcommand{\OperatorTok}[1]{\textcolor[rgb]{0.81,0.36,0.00}{\textbf{#1}}}
\newcommand{\OtherTok}[1]{\textcolor[rgb]{0.56,0.35,0.01}{#1}}
\newcommand{\PreprocessorTok}[1]{\textcolor[rgb]{0.56,0.35,0.01}{\textit{#1}}}
\newcommand{\RegionMarkerTok}[1]{#1}
\newcommand{\SpecialCharTok}[1]{\textcolor[rgb]{0.00,0.00,0.00}{#1}}
\newcommand{\SpecialStringTok}[1]{\textcolor[rgb]{0.31,0.60,0.02}{#1}}
\newcommand{\StringTok}[1]{\textcolor[rgb]{0.31,0.60,0.02}{#1}}
\newcommand{\VariableTok}[1]{\textcolor[rgb]{0.00,0.00,0.00}{#1}}
\newcommand{\VerbatimStringTok}[1]{\textcolor[rgb]{0.31,0.60,0.02}{#1}}
\newcommand{\WarningTok}[1]{\textcolor[rgb]{0.56,0.35,0.01}{\textbf{\textit{#1}}}}
\usepackage{graphicx}
\makeatletter
\def\maxwidth{\ifdim\Gin@nat@width>\linewidth\linewidth\else\Gin@nat@width\fi}
\def\maxheight{\ifdim\Gin@nat@height>\textheight\textheight\else\Gin@nat@height\fi}
\makeatother
% Scale images if necessary, so that they will not overflow the page
% margins by default, and it is still possible to overwrite the defaults
% using explicit options in \includegraphics[width, height, ...]{}
\setkeys{Gin}{width=\maxwidth,height=\maxheight,keepaspectratio}
% Set default figure placement to htbp
\makeatletter
\def\fps@figure{htbp}
\makeatother
\setlength{\emergencystretch}{3em} % prevent overfull lines
\providecommand{\tightlist}{%
  \setlength{\itemsep}{0pt}\setlength{\parskip}{0pt}}
\setcounter{secnumdepth}{-\maxdimen} % remove section numbering
\ifluatex
  \usepackage{selnolig}  % disable illegal ligatures
\fi

\author{}
\date{\vspace{-2.5em}}

\begin{document}

\hypertarget{create-your-first-data-frame}{%
\subsubsection{Create your first data
frame}\label{create-your-first-data-frame}}

\begin{Shaded}
\begin{Highlighting}[]
\NormalTok{animals}\OtherTok{=}\FunctionTok{c}\NormalTok{(}\StringTok{"Snake"}\NormalTok{,}\StringTok{"Ostrich"}\NormalTok{,}\StringTok{"Cat"}\NormalTok{,}\StringTok{"Spider"}\NormalTok{)}
\FunctionTok{print}\NormalTok{(}\FunctionTok{mode}\NormalTok{(animals))}
\end{Highlighting}
\end{Shaded}

\begin{verbatim}
## [1] "character"
\end{verbatim}

\begin{Shaded}
\begin{Highlighting}[]
\NormalTok{legs}\OtherTok{=}\FunctionTok{c}\NormalTok{(}\DecValTok{0}\NormalTok{,}\DecValTok{2}\NormalTok{,}\DecValTok{4}\NormalTok{,}\DecValTok{8}\NormalTok{)}
\FunctionTok{print}\NormalTok{(}\FunctionTok{mode}\NormalTok{(legs))}
\end{Highlighting}
\end{Shaded}

\begin{verbatim}
## [1] "numeric"
\end{verbatim}

\begin{Shaded}
\begin{Highlighting}[]
\NormalTok{df }\OtherTok{\textless{}{-}} \FunctionTok{data.frame}\NormalTok{(animals,legs)}
\NormalTok{df}
\end{Highlighting}
\end{Shaded}

\begin{verbatim}
##   animals legs
## 1   Snake    0
## 2 Ostrich    2
## 3     Cat    4
## 4  Spider    8
\end{verbatim}

\hypertarget{matrix-operations}{%
\subsubsection{Matrix operations}\label{matrix-operations}}

\begin{Shaded}
\begin{Highlighting}[]
\NormalTok{x\_vect }\OtherTok{\textless{}{-}} \FunctionTok{seq}\NormalTok{(}\DecValTok{12}\NormalTok{,}\DecValTok{2}\NormalTok{,}\SpecialCharTok{{-}}\DecValTok{2}\NormalTok{)}
\NormalTok{X }\OtherTok{\textless{}{-}} \FunctionTok{matrix}\NormalTok{(x\_vect,}\DecValTok{2}\NormalTok{,}\DecValTok{3}\NormalTok{)}
\NormalTok{X}
\end{Highlighting}
\end{Shaded}

\begin{verbatim}
##      [,1] [,2] [,3]
## [1,]   12    8    4
## [2,]   10    6    2
\end{verbatim}

\begin{Shaded}
\begin{Highlighting}[]
\NormalTok{Y }\OtherTok{\textless{}{-}} \FunctionTok{matrix}\NormalTok{(}\FunctionTok{seq}\NormalTok{(}\DecValTok{4}\NormalTok{),}\DecValTok{2}\NormalTok{,}\DecValTok{2}\NormalTok{)}
\NormalTok{Z }\OtherTok{\textless{}{-}} \FunctionTok{matrix}\NormalTok{(}\FunctionTok{seq}\NormalTok{(}\DecValTok{4}\NormalTok{,}\DecValTok{10}\NormalTok{,}\DecValTok{2}\NormalTok{),}\DecValTok{2}\NormalTok{,}\DecValTok{2}\NormalTok{)}
\NormalTok{Y}
\end{Highlighting}
\end{Shaded}

\begin{verbatim}
##      [,1] [,2]
## [1,]    1    3
## [2,]    2    4
\end{verbatim}

\begin{Shaded}
\begin{Highlighting}[]
\NormalTok{Z}
\end{Highlighting}
\end{Shaded}

\begin{verbatim}
##      [,1] [,2]
## [1,]    4    8
## [2,]    6   10
\end{verbatim}

\begin{Shaded}
\begin{Highlighting}[]
\FunctionTok{t}\NormalTok{(Y)}
\end{Highlighting}
\end{Shaded}

\begin{verbatim}
##      [,1] [,2]
## [1,]    1    2
## [2,]    3    4
\end{verbatim}

\begin{Shaded}
\begin{Highlighting}[]
\FunctionTok{t}\NormalTok{(Z)}
\end{Highlighting}
\end{Shaded}

\begin{verbatim}
##      [,1] [,2]
## [1,]    4    6
## [2,]    8   10
\end{verbatim}

\begin{Shaded}
\begin{Highlighting}[]
\NormalTok{Y}\SpecialCharTok{+}\NormalTok{Z}
\end{Highlighting}
\end{Shaded}

\begin{verbatim}
##      [,1] [,2]
## [1,]    5   11
## [2,]    8   14
\end{verbatim}

\begin{Shaded}
\begin{Highlighting}[]
\NormalTok{Z}\SpecialCharTok{+}\NormalTok{Y}
\end{Highlighting}
\end{Shaded}

\begin{verbatim}
##      [,1] [,2]
## [1,]    5   11
## [2,]    8   14
\end{verbatim}

\begin{Shaded}
\begin{Highlighting}[]
\NormalTok{Y}\SpecialCharTok{\%*\%}\NormalTok{Z}
\end{Highlighting}
\end{Shaded}

\begin{verbatim}
##      [,1] [,2]
## [1,]   22   38
## [2,]   32   56
\end{verbatim}

\begin{Shaded}
\begin{Highlighting}[]
\NormalTok{Z}\SpecialCharTok{\%*\%}\NormalTok{Y}
\end{Highlighting}
\end{Shaded}

\begin{verbatim}
##      [,1] [,2]
## [1,]   20   44
## [2,]   26   58
\end{verbatim}

\begin{Shaded}
\begin{Highlighting}[]
\NormalTok{Y}\SpecialCharTok{*}\NormalTok{Z}
\end{Highlighting}
\end{Shaded}

\begin{verbatim}
##      [,1] [,2]
## [1,]    4   24
## [2,]   12   40
\end{verbatim}

\begin{Shaded}
\begin{Highlighting}[]
\NormalTok{Z}\SpecialCharTok{*}\NormalTok{Y}
\end{Highlighting}
\end{Shaded}

\begin{verbatim}
##      [,1] [,2]
## [1,]    4   24
## [2,]   12   40
\end{verbatim}

\begin{Shaded}
\begin{Highlighting}[]
\NormalTok{Y}\SpecialCharTok{\%*\%}\NormalTok{X}
\end{Highlighting}
\end{Shaded}

\begin{verbatim}
##      [,1] [,2] [,3]
## [1,]   42   26   10
## [2,]   64   40   16
\end{verbatim}

\begin{Shaded}
\begin{Highlighting}[]
\FunctionTok{solve}\NormalTok{(Y)}
\end{Highlighting}
\end{Shaded}

\begin{verbatim}
##      [,1] [,2]
## [1,]   -2  1.5
## [2,]    1 -0.5
\end{verbatim}

\begin{Shaded}
\begin{Highlighting}[]
\FunctionTok{solve}\NormalTok{(Y)}\SpecialCharTok{\%*\%}\NormalTok{X}
\end{Highlighting}
\end{Shaded}

\begin{verbatim}
##      [,1] [,2] [,3]
## [1,]   -9   -7   -5
## [2,]    7    5    3
\end{verbatim}

\hypertarget{writing-a-simple-function-within-r}{%
\subsubsection{Writing a simple function within
R}\label{writing-a-simple-function-within-r}}

\begin{Shaded}
\begin{Highlighting}[]
\NormalTok{myFirstRFunc }\OtherTok{\textless{}{-}} \ControlFlowTok{function}\NormalTok{(x)\{}
    \FunctionTok{stopifnot}\NormalTok{(}\FunctionTok{is.numeric}\NormalTok{(x),x}\SpecialCharTok{\%\%}\DecValTok{1} \SpecialCharTok{==} \DecValTok{0}\NormalTok{,x}\SpecialCharTok{\textgreater{}}\DecValTok{1}\NormalTok{)}
\NormalTok{    sum }\OtherTok{\textless{}{-}} \DecValTok{0}
    \ControlFlowTok{for}\NormalTok{(i }\ControlFlowTok{in} \DecValTok{1}\SpecialCharTok{:}\NormalTok{(x}\DecValTok{{-}1}\NormalTok{))\{}
        \ControlFlowTok{if}\NormalTok{((i}\SpecialCharTok{\%\%}\DecValTok{7}\SpecialCharTok{==}\DecValTok{0}\NormalTok{)}\SpecialCharTok{||}\NormalTok{(i}\SpecialCharTok{\%\%}\DecValTok{2}\SpecialCharTok{==}\DecValTok{0}\NormalTok{))\{}
\NormalTok{            sum}\OtherTok{=}\NormalTok{ sum}\SpecialCharTok{+}\NormalTok{i}
\NormalTok{        \}}
\NormalTok{    \}}
    \FunctionTok{return}\NormalTok{(sum)}
\NormalTok{\}}
\FunctionTok{myFirstRFunc}\NormalTok{(}\DecValTok{14}\NormalTok{)}
\end{Highlighting}
\end{Shaded}

\begin{verbatim}
## [1] 49
\end{verbatim}

\begin{Shaded}
\begin{Highlighting}[]
\FunctionTok{myFirstRFunc}\NormalTok{(}\DecValTok{1000}\NormalTok{)}
\end{Highlighting}
\end{Shaded}

\begin{verbatim}
## [1] 284787
\end{verbatim}

\hypertarget{generating-an-r-markdown-notebook}{%
\subsubsection{Generating an R markdown
notebook}\label{generating-an-r-markdown-notebook}}

\hypertarget{inclucing-blocks-of-code}{%
\paragraph{Inclucing blocks of code}\label{inclucing-blocks-of-code}}

\begin{Shaded}
\begin{Highlighting}[]
\NormalTok{x }\OtherTok{\textless{}{-}} \FunctionTok{seq}\NormalTok{(}\DecValTok{0}\NormalTok{,}\DecValTok{20}\NormalTok{,}\FloatTok{0.01}\NormalTok{)}
\NormalTok{y }\OtherTok{\textless{}{-}} \FunctionTok{sin}\NormalTok{(x)}
\NormalTok{sin\_df }\OtherTok{\textless{}{-}} \FunctionTok{data.frame}\NormalTok{(x,y)}
\FunctionTok{head}\NormalTok{(sin\_df,}\DecValTok{3}\NormalTok{)}
\end{Highlighting}
\end{Shaded}

\begin{verbatim}
##      x           y
## 1 0.00 0.000000000
## 2 0.01 0.009999833
## 3 0.02 0.019998667
\end{verbatim}

\hypertarget{including-plots}{%
\paragraph{Including plots}\label{including-plots}}

\begin{Shaded}
\begin{Highlighting}[]
\FunctionTok{plot}\NormalTok{(x,y)}
\end{Highlighting}
\end{Shaded}

\includegraphics{test_files/figure-latex/unnamed-chunk-8-1.pdf} \#\#\#\#
Displaying mathematical formulae \(y=\sin(x)\)
\(\[ \sin(x)=\sum_{n=1}^{\infty}(-1)^{n+1}\cdot \frac{x^{2n-1}}{(2n-1)!}\approx x-\frac{x^3}{3!}+\frac{x^5}{5!}-\frac{x^7}{7!}\ldots. \]\)
\(\[ A=\left(\begin{matrix} a_{11} & a_{12}\\ a_{21} & a_{22} \end{matrix}\right). \]\)

\end{document}
